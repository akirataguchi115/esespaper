\documentclass[conference]{IEEEtran}
\usepackage{cite}
\usepackage{amsmath,amssymb,amsfonts}
\usepackage{algorithmic}
\usepackage{graphicx}
\usepackage{textcomp}
\usepackage{xcolor}
\usepackage{url}
\def\BibTeX{{\rm B\kern-.05em{\sc i\kern-.025em b}\kern-.08em
    T\kern-.1667em\lower.7ex\hbox{E}\kern-.125emX}}
\begin{document}

\title{Differences between open source and free software}

\author{\IEEEauthorblockN{Akira Taguchi}
\IEEEauthorblockA{\textit{Department of Computer Science} \\
\textit{University of Helsinki}\\
Finland\\
akirahattivatti@gmail.com ORCID: 0009-0007-0343-4540}
}

\maketitle

\begin{abstract}
Open-source software development is praised by an increasing amount of big technology companies. The outcome of open-source development is however problematic to the developers themselves. Free software attempts to solve these problems. This paper examines the differences between open source and free software definitions and their usage. The results show some examples of the differences. The paper then attempts to go through some of the phenoma behind the unclear differences. Both open source and free software share some of the same development methods. Free software however doesn't share all of the same sustainability issues open source does.
\end{abstract}

\begin{IEEEkeywords}
	open-source software, open source, free software, software freedom, libre software, proprietary software, closed source software, copyleft
\end{IEEEkeywords}

\section{Introduction}
Open-source software development has seen an increasing interest in the recent years \cite{openvsclosed}. Open source gives all kinds of different benefits ranging from enhanced security to labor force from outside the company. It's most known oppositor, closed-source software, comes with it's own benefits like increased revenue because nobody can create a competitive copy with the same source code. Free software, which came before open source, has seen declined interest in the last four decades \cite{declinewiki}. Although open-source software is often referred as "Free and Open Source Software", or FLOSS, Free software wants to stand out deliberately from open source \cite{stallman2002free}.

The paper structure follows the IMRAD structure. First we take a look at the methods we're going to use for defining open source and free software. These methods contain looking up definitions from Open Source Initiative, FSF and other more cited sources. Next in results we will go over some phenomena in detail that are directly bound to the differences of open source and free software. This phenoma include a new term "openwashing" and the demonization of free software. Lastly we will go through some next actions to take regarding open source versus free software.

\section{Methods}

\section{Results}

\section{Discussion}

\bibliographystyle{IEEEtran}
\bibliography{refs}{}

\end{document}
