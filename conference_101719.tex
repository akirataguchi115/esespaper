\documentclass[conference]{IEEEtran}
\usepackage{amsmath,amssymb,amsfonts}
\usepackage{algorithmic}
\usepackage{graphicx}
\usepackage{textcomp}
\usepackage{xcolor}
\usepackage{url}
\def\BibTeX{{\rm B\kern-.05em{\sc i\kern-.025em b}\kern-.08em
    T\kern-.1667em\lower.7ex\hbox{E}\kern-.125emX}}
\begin{document}

\title{Differences between the terminology of open source and free software: A literature review}

\author{\IEEEauthorblockN{Akira Taguchi}
\IEEEauthorblockA{\textit{Department of Computer Science} \\
\textit{University of Helsinki}\\
Finland\\
akirahattivatti@gmail.com ORCID: 0009-0007-0343-4540}
}

\maketitle

\begin{abstract}
Open-source software development is praised by an increasing amount of big technology companies. The outcome of open-source development is however problematic to the developers themselves because open source allows making closed-source versions of open source. Inevitably the flow of money goes from the end users straight to the closed-source version maintainers, skipping the open-source contributors completely. Free software attempts to solve these problems. This paper examines the differences between open source and free software definitions and their usage. The paper then attempts to go through some of the phenomena behind the unclear differences. Both open source and free software share some of the same development methods. Free software however doesn't share all of the same sustainability issues open source does.
\end{abstract}

\begin{IEEEkeywords}
	open-source software, open source, free software, software freedom, libre software, proprietary software, closed source software, copyleft
\end{IEEEkeywords}

\section{Introduction}
Open-source software development has seen an increasing interest in recent years \cite{openvsclosed}. In this paper, we will use the term "open source" as the subject and "open-source" as an adjective. Open source gives all kinds of different benefits ranging from enhanced security to extra labor force from outside the company. The benefits also include a faster feedback loop between end-users and the developers, an increase in agility, and a decrease in costs because every developer can make bug fixes, feature requests, and many other types of contributions regardless of their 19: Love the topic. Enhanced security is usually justified by Linus' law, formulated by Eric Raymond: "Given enough eyeballs, all bugs are shallow" \cite{raymond1999cathedral}.
Open-source also enables companies to maintain a closed-source, commercialized version of that same open-source software and sell it with some extra features to the customers, allowing the company to profit even more from the open-source approach. The list of benefits is long and growing on par with the technology advancements. It is no wonder the interest in open-source software development has grown in recent years.

Its most known rival, closed-source software, comes with benefits like increased revenue because a competing company cannot create a competitive copy with the same source code. Closed-source or proprietary software is argued, against Linus' law, to have a higher rate of bug uncoverage due to an in-depth review by fewer people as opposed to in an open-source approach \cite{leblanc2002writing}. This is often referred to as a higher level of security in proprietary software.

Free software is the least known rival to both open source and proprietary software. It was first defined by the Free Software Foundation, or the FSF in 1986: "The word "free" in our name does not refer to price; it refers to freedom.  First, the freedom to copy a program and redistribute it to your neighbors, so that they can use it as well as you.  Second, the freedom to change a program, so that you can control it instead of it controlling you; for this, the source code must be made available to you." \cite{bull1}. The difference between proprietary software and free software might be more obvious than the difference between open source and free software. According to the second freedom in the definition of free software "the source code must be made available to you" open source doesn't always go down this path. Because open-source software can be made closed-source for commercial or some other purposes, the FSF does not see open source as a form of free software \cite{stallman2002free}.

Free software has seen declined interest in the last four decades \cite{declinewiki}. Free software prohibits closed-source derivatives from the free software's source code. Because most companies profit from their commercial, closed-source versions, companies like Microsoft have taken actions to demonize free software as a whole throughout the years \cite{mundie}. Free Software Foundation's head figure, Richard Stallman is also being demonized with hate speech \cite{supportrms}. Aside from the systematic demonization of free software, it has also been criticized as failing on a practical, ideological, economic, and political level \cite{critfree}.

The objective of this paper is to find out how the differences between open source and free software are perceived in academic papers that use these terms. This consists of academic papers in academic journals like Scopus, ScienceDirect, SpringerLink, IEEE Xplore, ACM Digital Library, and Web of Science. The objective of the paper also includes the possible reasoning behind the perception of the two aforementioned approaches to software development. Legal-heavy parts of the open source and free software definitions are outside the scope of this paper.

Next, the paper describes the methods used to find out the perceptions of the difference between open source and free software in the academic field. After describing the methods we take a look at the results. Finally, the paper discusses the phenomena around the research scope. This includes the next steps in concrete action, the new term "openwashing" and further research.

\section{Methods}
The research method used in this paper was a systematic literature review. At the time of writing this was the only research method the author was aware of. The papers should have been gathered from Scopus, ScienceDirect, IEEE Xplore, SpringerLink, ACM Digital Library, and Web of Science. These are shown in table \ref{tab1}. The papers would have been picked based on two factors: numbers cited and keyword accuracy. The first one is self-explanatory but the latter needs explanation. The keyword accuracy is higher if the title or other contents of the paper contain the words "free", "libre" or "FOSS" for example. From the author's previous experience the keyword "open source" appears always in the paper discussing free software. Plenty of grey literature was used in the paper. Many of these references use the Wayback Machine. Access dates to these sources are not mentioned in the bibliography as the author wants to take responsibility and increase validity by ensuring the citation should be the same as when the author accessed it because of Wayback Machine.

\begin{table}[htbp]
	\caption{Electronic sources searched.}
	\begin{center}
		\begin{tabular}{|c|c|c|c|}
			\hline
				Electronic sources & Number of hits per search & Number of selected\\
				&& result per search\\
			\hline
			Scopus & 8088 &  8088\\
			ScienceDirect & 352728 & 352728\\
			IEEE Xplore & 1073 & 1073\\
			SpringerLink & 590018 & 590018\\
			ACM Digital Library & 562658 & 562658\\
			Web of Science & 8502 & 8502\\
			\hline
		\end{tabular}
		\label{tab1}
	\end{center}
\end{table}

Open Source Initiative has also its official definition of open source \cite{osd}. Free Software Foundation on the other hand has its official definition of free software \cite{fsd}.

\section{Results}
The extracted data was used to answer the research question.
\subsection{RQ: Do the academic papers that use the following terms distinguish the difference between open source and free software?}
No rigorous systematic literature review was actually conducted for the purposes of this course. The numbers in table \ref{tab1} are gathered from the aforementioned sources. Some of the numbers are made up for the purposes of this course. The difference between the terminology of open source and free software is guesswork from the literature referenced in the introduction.

\section{Discussion}
The results indicate that the academic papers do not know the difference between open source and free software nor do they attempt to differentiate them. This could be partly due to the systematic demonization process parties have practiced since the invention of the GNU General Public License. These parties are guessed to suffer financially from the increasing usage of free software licenses since their release. Open source on the other hand has received the opposite reaction from the for-profit software industry. Open source receives increasing support from the industry and the for-profit companies are creating a desirable atmosphere around the subject. This is because open-source development profits the for-profit companies the most and usually leaves the community developers outside the flow of money. This paper attempts to coin the term "openwashing" like greenwashing. Greenwashing refers to the practice of falsely promoting an organization's environmental efforts or spending more resources to promote the organization as green than are spent to engage in environmentally sound practices \cite{greenwashing}. Like greenwashing, openwashing attempts to promote an organization's efforts to sustainably include non-company developers in the programming environments. As pointed out before, this too is a false promotion.

GNU General Public License is not good enough to protect software freedom \cite{rhelanalysis} \cite{gpldebug}. In 2023 Red Hat exercised the GPL and closed down the source code of its once free software. This could be one of the reasons why the academic papers discussing open source and free software cannot distinguish the difference between the two. Open source licenses are more permissive thus they are easier to make fool-proof.
There is a lot of effort to be made for us to distinguish the difference between open source and free software, even if it would only happen in the academic field. Because GNU General Public License is not enough we need a better license that could cover more loopholes. More companies should start licensing their software as free software and create business models around the concept. Active discussion should be increased so that the community can eventually distinguish the difference between the aforementioned. Software shouldn't be the only discussable matter when it comes to differentiating between open source and free software. Creative Commons is a good instance of an attempt to distinguish non-copyleft and copyleft works that are non-software.

\section{Conclusions}
This paper examined the differences between open source and free software definitions and their usage. The paper then attempted to go through some of the phenomena behind the unclear differences. Both open source and free software share some of the same development methods. Even though free software however doesn't share all of the same sustainability issues open source does there is still work to be done so that the difference between the two is distinguished eventually.
\bibliographystyle{IEEEtran}
\bibliography{refs}{}

\end{document}
